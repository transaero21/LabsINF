\section{Описание использованных типов данных}

При выполнении лабораторной работы были использованный следующие типы данных:
\begin{enumerate}
    \item Структура \texttt{Student} студента по заданию, имеющие следующие поля:
    \texttt{.name} - указатель (\texttt{char*}) на имя студента,
    \texttt{.group} - номер (\texttt{int}) студента, \texttt{.score} - средний балл (\texttt{float}) студента.
    \item Для удобного хранения данных о массиве была создана структура \texttt{Students},
    имеющие следующие поля: \texttt{.ptr} - указатель (\texttt{Student*}) на массив,
    \texttt{.size} - длина (\texttt{int}) массива.
    \item Для удобного хранения данных о файлах была создана структура \texttt{Files},
    имеющие следующие поля: \texttt{.pathIn} - указатель (\texttt{char*}) на строку исходного файла,  
    \texttt{.pathOut} - указатель (\texttt{char*}) на строку итогового файла.
    \item Для удобного хранения данных о  была создана структура \texttt{RandParams},
    имеющие следующие поля: \texttt{.tries} - количество (\texttt{int}) потворений сортировки,  
    \texttt{.size} - размер (\texttt{int}) тестируемого массива.
\end{enumerate}
