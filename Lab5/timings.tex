\section{Анализ результатов таймирования}

\pgfplotsset{compat=1.9}
\pgfplotsset{
    qsort/.style = {green},
    odd-even/.style = {blue},
    double-selection/.style = {red}
}

На графике представлено среднее время сортировок в зависимости от количества сортируемых элементов.
Для наиболее точного подсчёта среднего времени количество повторений сортировок было 30 раз.
Такое среднее время находилось для каждого поля отдельно, а затем вычислялось среднее время.
Так было подсчитано среднее время для всех сортировок от 50 до 10000 элементом с шагом в 10 элементов.

\begin{figure}[H]
    \centering
    \begin{tikzpicture}
        \begin{axis}[
                legend pos=north west,
                axis lines=left,
                width=0.725\textwidth,
                xmin=50,
                xmax=10000,
                xtick={0,1000,...,10000},
                scaled x ticks = false,
                xticklabel style={rotate=90},
                log ticks with fixed point,
                xlabel=$n$,
                ymin=0,
                ymax=0.8,
                ytick={0,0.1,...,1},
                ylabel=$t(n)$,
                grid=major
            ]
            \legend{qsort, odd-even, double-selection};
            \addplot[qsort] table [x=size, y=qsort, col sep=comma] {src/timings.csv};
            \addplot[odd-even] table [x=size, y=odd-even, col sep=comma] {src/timings.csv};
            \addplot[double-selection] table [x=size, y=double-selection, col sep=comma] {src/timings.csv};
        \end{axis}
    \end{tikzpicture}
    \caption{Общий график сортировок.}
\end{figure}

Для анализа результата графика приведём среднюю сложность алгоритмов сортировок:

\begin{itemize}
    \item \texttt{odd-even} - $O(n^2)$.
    \item \texttt{double-selection} - $O(n^2 / 2)$.
    \item \texttt{qsort} - $O(n * \log n)$.
\end{itemize}

Как видно, сортировки \texttt{odd-even} и \texttt{double-selection} имеют параболический вид, 
а \texttt{double-selection} имеет результаты времени в два раза меньшие чем у \texttt{odd-even},
что и понятно из приведённого списка средней сложноси выше. Из-за низкой сложность алгоритма \texttt{qsort}
на графике в сравнении с остальными сортировки его результаты стремятся к нулю.

\begin{figure}[H]
    \centering
    \begin{tikzpicture}
        \begin{axis}[
                legend pos=north east,
                axis lines=left,
                width=0.725\textwidth,
                xmin=50,
                xmax=10000,
                xtick={0,1000,...,10000},
                scaled x ticks = false,
                xticklabel style={rotate=90},
                xlabel=$n$,
                ymin=0,
                ymax=0.015,
                scaled y ticks = false,
                ytick={0,0.0015,...,0.015},
                ylabel=$t(n)$,
                log ticks with fixed point,
                grid=major
            ]
            \legend{qsort, odd-even, double-selection};
            \addplot[qsort] table [x=size, y=qsort, col sep=comma] {src/timings.csv};
            \addplot[odd-even] table [x=size, y=odd-even, col sep=comma] {src/timings.csv};
            \addplot[double-selection] table [x=size, y=double-selection, col sep=comma] {src/timings.csv};
        \end{axis}
    \end{tikzpicture}
    \caption{Увеличенный график сортировок.}
\end{figure}

На увеличенном графике можно увидеть насколько скорость \texttt{qsort} отличается от скорости
остальных сортировок. Таким образом можно сделать вывод, что из тестируемых сортировок
самая быстрая - \texttt{qsort}, а медленная - \texttt{odd-even}.

