\section{Описание тестовых примеров}

Важно отметить, что комманды в последовательности разделены запятыми,
который являются Enter или Return, а эелементы массива также разделены
запятыми, однако уже как отдельные элементы. Необходимые Enter для продолжения
и выход из программы в последовательности команд опущены.

\begin{table}[H]
  \centering
  \begin{tblr}{|Q[c,5.5cm]|Q[c,5.5cm]|Q[c,5.5cm]|}
    \hline
    Последовательность комманд &
    Ожидаемый массив &
    Полученный массив \\
    \hline
    1, 7, 3, 6, 8, 4, 6, 3, 2, 13, 111, 3, 1  &
    1, 6, 8, 4, 6, 3, 0, 0, 0, 0, 0, 0, 111 &
    1, 6, 8, 4, 6, 3, 0, 0, 0, 0, 0, 0, 111 \\
    \hline
    1, 10, 123, -468, 43, 0, 1, 45, 1235, 975, 23, 1, 4, 2, 6, 423 &
    123, -468, 975, 0, 0, 0, 423 &
    123, -468, 975, 0, 0, 0, 423 \\
    \hline
    1, 13, 2356, 211, 542, 8, 3, 1, 2, 3, 4, 4563, 234, 897, 3, 4, 3, 0, 4, 2 &
    234 &
    234 \\
    \hline
    1, 21, 2, 53, 6, 76, 45, 8, 5, 2, 4, 67, 854, 3, 4123, 45, -9, -43, 23402395, 952, 528, 394, 505, 4, 2 &
    Пустой массив &
    Пустой массив \\
    \hline
    1, 5, 1, 2, 3, 4, 5, 2, 6, 123, 3, 0, 2, 2, 456, 4, 1&
    2, 3, 4, 5, 0 &
    2, 3, 4, 5, 0 \\
    \hline
  \end{tblr}
  \caption{Тестовые примеры}
\end{table}

