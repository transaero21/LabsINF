\newpage
\section{Исходные коды разработанных программ}

% Makefile
\subsection{Исходный код конфигурации компиляции}
\lstinputlisting[
  caption={файл конфигурации \texttt{Makefile}},
  language=make,
]{Makefile}

% main.c
\subsection{Исходный код первоначальной инициализации}
\lstinputlisting[
  caption={библиотека \texttt{main.c}},
  language=C,
]{main.c}

% menu.c / include/menu.h
\subsection{Исходный код главного меню}
\lstinputlisting[
  caption={библиотека \texttt{menu.c}.},
  language=C,
]{menu.c}
\lstinputlisting[
  caption={заголовочный файл \texttt{include/menu.h}.},
  language=C,
]{include/menu.h}

% arrays.c / include/arrays.h
\subsection{Исходный код обработки массивов}
\lstinputlisting[
  caption={библиотека \texttt{arrays.c}.},
  language=C,
]{arrays.c}
\lstinputlisting[
  caption={заголовочный файл \texttt{include/arrays.h}.},
  language=C,
]{include/arrays.h}

% input.c / include/input.h
\subsection{Исходный код обработки ввода пользователя}
\lstinputlisting[
  caption={библиотека \texttt{input.c}.},
  language=C,
]{input.c}
\lstinputlisting[
  caption={заголовочный файл \texttt{include/input.h}.},
  language=C,
]{include/input.h}

% utils.c / include/utils.h
\subsection{Исходный код дополнительных вспомогательных функций}
\lstinputlisting[
  caption={библиотека \texttt{utils.c}.},
  language=C,
]{utils.c}
\lstinputlisting[
  caption={заголовочный файл \texttt{include/utils.h}.},
  language=C,
]{include/utils.h}