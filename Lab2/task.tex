\section{Формулировка индивидуального задания}

\textbf{Вариант №74:} Вычислить значение функции в точке при помощи разложения в ряд:
\begin{equation}
\sin^3 x = \frac{3}{4} (\frac{8x^3}{3!} - \frac{80x^5}{5!} +\
\frac{728x^7}{7!} - ...) = \frac{3}{4}\sum\limits_{n=1}^\infty\
(-1)^{n+1}\frac{3^{2n} - 1}{(2n+1)!}x^{2n+1}, |x| < \infty
\end{equation}