\section{Описание тестовых примеров}

\subsection{Тестовые примеры для подсчёта по количеству членов ряда}

\begin{table}[H]
  \centering
  \begin{tabular}{| c | c | c | c |}
    \hline
    Аргумент функции & Количество членов & Значение с \texttt{math.h} & Полученное значение \\
    \hline
    1.4 & 20 & 0.9569812287889454 & 0.9569808840751648 \\
    \hline
    -1 & 10 & -0.5958232365909556 & -0.5958232283592224 \\
    \hline
    5 & 40 & -0.8817651660366330 & -0.8916047811508179 \\
    \hline
    -3.17263 & 200 & 0.0000298846083317 & 0.0000606863395660 \\ 
    \hline
    0 & 100 & 0.0000000000000000 & 0.0000000000000000 \\
    \hline
  \end{tabular}
  \caption{Тестовые примеры (подсчёт по количеству членов ряда)}
\end{table}

\subsection{Тестовые примеры для подсчёта по заданной точности}

\begin{table}[H]
  \centering
  \begin{tabular}{| c | c | c | c |}
    \hline
    Аргумент функции & Заданная точность & Значение с \texttt{math.h} & Полученное значение \\
    \hline
    1 & 0.1 & 0.5958232365909556 & 0.5947751402854919 \\
    \hline
    -0.9 & 0.0003 & -0.4806501848808332 & -0.4806496202945709 \\
    \hline
    4.12937 & 0.0009 & -0.5817740149783298 & -0.5823402404785156 \\
    \hline
    1.234567 & 0.9 & 0.8412469450644317 & 0.9212406277656555 \\
    \hline
    3.21 & 0.0000001 & -0.0003193689160720 & -0.0003066345525440 \\
    \hline
  \end{tabular}
  \caption{Тестовые примеры (подсчёт по заданной точности)}
\end{table}
